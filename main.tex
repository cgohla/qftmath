%! TeX program = lualatex
\documentclass[a4paper, 12pt, oneside]{memoir}

\setulmarginsandblock{3cm}{3cm}{*}
\setlrmarginsandblock{3cm}{3cm}{*}
\checkandfixthelayout

% Style
\chapterstyle{dash}
\renewcommand*{\chaptitlefont}{\normalfont\Huge\bfseries\sffamily}
\renewcommand*{\chapnamefont}{\normalfont\huge\bfseries\sffamily}
\renewcommand*{\chapnumfont}{\normalfont\huge\bfseries\sffamily}
\setsecheadstyle{\normalfont\Large\bfseries\sffamily}
%\renewcommand{\chaptername}{Lecture}

% Tikz
\usepackage{tikz}
\usetikzlibrary{cd}

% Bibliography
\usepackage{hyperref}
\usepackage[backend=biber, style=alphabetic]{biblatex}
\addbibresource{library.bib}


% Tables
\usepackage{booktabs}
\usepackage{makecell}

% Typesetting
\usepackage[english]{babel}
\usepackage{csquotes}
\usepackage{libertinus-otf}
%\OnehalfSpacing

% Theorem environments
\usepackage{amsthm}
\usepackage{thmtools}
\declaretheorem[style=definition]{definition}
\declaretheorem[style=definition, numbered=no]{example}
\declaretheorem[style=definition, numbered=no]{exercise}


% Symbols
\usepackage{slashed}

% Macros
\newcommand{\Ecal}{\mathcal{E}}
\newcommand{\Fcal}{\mathcal{F}}
\newcommand{\Lcal}{\mathcal{L}}
\newcommand{\Ocal}{\mathcal{O}}
\newcommand{\Xfrak}{\mathfrak{X}}
\newcommand{\gfrak}{\mathfrak{g}}
\newcommand{\Rbb}{\mathbb{R}}
\newcommand{\Lbb}{\mathbb{L}}
\newcommand{\Pbb}{\mathbb{P}}
\newcommand{\drm}{\mathrm{d}}
\newcommand{\Hrm}{\mathrm{H}}
\newcommand{\dd}{\, \mathrm{d}}
\newcommand{\hodge}{{\star}}
\newcommand{\bigslant}[2]{{\left.\raisebox{.2em}{$#1$}\middle/\raisebox{-.2em}{$#2$}\right.}}
\DeclareMathOperator{\tr}{tr}
\DeclareMathOperator{\Crit}{Crit}
\DeclareMathOperator{\dCrit}{dCrit}
\DeclareMathOperator{\Graph}{Graph}
\DeclareMathOperator{\PV}{PV}
\DeclareMathOperator{\Sym}{Sym}

\title{Quantum Field Theory for Mathematicians}
\author{}
\date{}

\begin{document}
\maketitle

\chapter{Lecture 1}

\section{History}

\begin{center}
  \begin{tabular}{ccccc}
    Date & People & What & Why & Techniques \\
    \midrule
    1969 &
    \makecell{ Faddeev \\ and Popov } &
    \makecell{ Gauge fixing \\ (adding ghosts) } &
    \makecell{ Quantize \\ Yang-Mills } &
    \makecell{ Berezinian \\ integration } \\
    \addlinespace
    1973 & 
    \makecell{ 't Hooft and \\ Veltman } &
    \makecell{ Quantized \\ Yang-Mills } &
    \makecell{ Quantize \\ Yang-Mills } &
    \makecell{ Feynman \\ diagrams } \\
    \addlinespace
    1975 &
    \makecell{ Becchi, Rouet, \\ Stora, Tyutin \\ (BRST) } &
    \makecell{ Cohomological \\ theory to quantize \\ Yang-Mills } &
    \makecell{ Understanding \\ 't Hooft \\ and Veltman } &
    \makecell{ Derived invariants \\ (Lie algebra \\ cohomology) } \\
    \addlinespace
    1981 &
    \makecell{ Batallin and \\ Vilkovisky \\ (BV) } &
    \makecell{ Quantize systems \\ with complicated \\ gauge symmetries } &
    Supergravity &
    \makecell{ Derived \\ intersections \\ (Koszul complexes) } \\
    \addlinespace
    1992 & Henneaux &
    \makecell{ Quantize \\ Yang-Mills \\ using BV } &
    \makecell{ Analyze \\ Yang-Mills \\ using BV } &
    \makecell{ Derived \\ intersections} \\
    \addlinespace
    2007 & Costello &
    \makecell{ Combine BV \\ with effective \\ field theory } &
    \makecell{ Make BV \\ quantization \\ rigorous } &
    \makecell{ Derived everything, \\ analysis, and \\ homotopy theory }
  \end{tabular}
\end{center}

\section{References}

The main references for this seminar will be:

\begin{itemize}
  \item Costello - Renormalization and Effective Field Theory \cite{costelloRenormalizationEffective};
  \item Elliot, Williams, Yoo - Asymptotic Freedom in the BV Formalism \cite{elliottAsymptoticFreedom};
  \item Gwilliam - Factorization algebras and free field theories \cite{gwilliamFactorizationAlgebras}.
\end{itemize}

\section{Roadmap to BV Quantization}

\begin{figure}
  \begin{tikzpicture}
  \draw[thick, domain=-7:7, smooth, variable=\x] plot ( {\x}, {0.15*sin(deg(\x+2.2))} );
  \node at (-6, 1.2) {
    \begin{tabular}{c}
      Classical \\
      BV theory
    \end{tabular}
  };
  \draw[thick, ->] (-6, .7) -- (-6, .2);
  \node at (-3, 1) {
    \begin{tabular}{c}
      Quantum \\
      BV theory
    \end{tabular}
  };
  \draw[thick, ->] (-3, .5) -- (-3, 0);
  \node at (0, 1) { Quantization };
  \draw[thick, ->] (0, .7) -- (0, .2);
  \node at (2.9, 1.6) {
    \begin{tabular}{c}
      Functional \\
      analysis
    \end{tabular}
  };
  \node at (1.2, .4) { \miniscule\texttt{DANGER!} };
  \draw[thick] (.7, .25) rectangle ++(.95, .3);
  \draw[thick] (1.2, .25) -- (1.2, -.03);
  \node at (6, 1.2) {
    \begin{tabular}{c}
      Quantum \\ 
      Yang-Mills
    \end{tabular}
  };
  \draw[thick, ->] (6, .7) -- (6, .2);
  % Mountains
  \draw[thick] (1.48, -.08) -- (1.75, 0.3);
  \draw[thick] (1.75, 0.3) -- (2.03, -.13);
  \draw[fill] (1.75, 0.3) circle (.2pt);
  \draw[semithick] (1.67, .2) -- (1.83, 0.19);
  \draw[thick] (1.87, .12) -- (2.15, 0.67);
  \draw[thick] (2.15, 0.67) -- (2.52, -.14);
  \draw[fill] (2.15, 0.67) circle (.19pt);
  \draw[semithick] (2.03, 0.43) -- (2.12, 0.5);
  \draw[fill] (2.12, 0.5) circle (.101pt);
  \draw[semithick] (2.12, 0.5) -- (2.19, 0.44);
  \draw[fill] (2.19, 0.44) circle (.101pt);
  \draw[semithick] (2.19, 0.44) -- (2.24, 0.47);
  \draw[thick] (2.28, .4) -- (2.45, 0.75);
  \draw[thick] (2.45, 0.75) -- (2.9, -.15);
  \draw[fill] (2.45, 0.75) circle (.19pt);
  \draw[semithick] (2.36, 0.58) -- (2.42, 0.55);
  \draw[fill] (2.42, 0.55) circle (.101pt);
  \draw[semithick] (2.42, 0.55) -- (2.48, 0.58);
  \draw[fill] (2.48, 0.58) circle (.101pt);
  \draw[semithick] (2.48, 0.58) -- (2.55, 0.55);
  \draw[thick] (2.53, .58) -- (2.65, 0.8);
  \draw[thick] (2.65, 0.8) -- (2.79, .57);
  \draw[fill] (2.65, 0.8) circle (.19pt);
  \draw[thick] (2.66, .32) -- (3, 1);
  \draw[thick] (3.65, -.06) -- (3, 1);
  \draw[fill] (3, 1) circle (.19pt);
  \draw[semithick] (2.89, .8) -- (2.95, .75);
  \draw[fill] (2.95, 0.75) circle (.101pt);
  \draw[semithick] (2.95, .75) -- (3, .8);
  \draw[fill] (3, 0.8) circle (.101pt);
  \draw[semithick] (3, .8) -- (3.05, 0.75);
  \draw[fill] (3.05, 0.75) circle (.101pt);
  \draw[semithick] (3.05, 0.75) -- (3.12, 0.82);
  \draw[thick] (3.3, 0.5) -- (3.45, 0.85);
  \draw[thick] (3.45, 0.85) -- (3.9, -.03);
  \draw[fill] (3.45, 0.85) circle (0.19pt);
  \draw[semithick] (3.36, .65) -- (3.43, .69);
  \draw[fill] (3.43, .69) circle (.101pt);
  \draw[semithick] (3.43, .69) -- (3.49, .65);
  \draw[fill] (3.49, .65) circle (.101pt);
  \draw[semithick] (3.49, .65) -- (3.54, .69);
  \draw[thick] (3.67, .4) -- (3.85, .65);
  \draw[thick] (3.85, 0.65) -- (4.2, 0.02);
  \draw[fill] (3.85, 0.65) circle (.19pt);
\end{tikzpicture}
\centering
\caption{Roadmap to BV quantization.}
\label{fig:roadmap}
\end{figure}

The space of fields $\Ecal^\bullet$ is a cochain complex
\begin{equation*}
  \begin{tikzcd}
    \dots \arrow[r] &
    \Ecal^{-1} \arrow[r, "Q"] &
    \Ecal^0 \arrow[r, "Q"] &
    \Ecal^1 \arrow[r, "Q"] &
    \Ecal^2 \arrow[r] &
    \dots
  \end{tikzcd}
\end{equation*}
equipped with a differential $Q$ such that $Q^2 = 0$. Moreover, $\Ecal$ admits a $-1$-shifted symplectic structure, that is, there exists a non degenerate pairing of degree $-1$
\begin{equation*}
  \langle \cdot , \cdot \rangle :
  \Ecal \otimes \Ecal \longrightarrow \Rbb [-1]
\end{equation*}
such that $\langle x, y \rangle = -(-1)^{(|x|+1)(|y|+1)} \langle y, x \rangle$. This structure defines a $+1$-shifted Poisson bracket
\begin{equation*}
  \{ \cdot, \cdot \} :
  \Ocal( \Ecal ) \otimes \Ocal( \Ecal ) \longrightarrow \Ocal( \Ecal )
\end{equation*}
where $\Ocal ( \Ecal ) \cong \mathrm{Sym}^\bullet (\Ecal^\vee)$ is the (graded) commutative algebra of polynomial functions on the dual complex $\Ecal^\vee$. Pick $S \in \Ocal (\Ecal)$ obeying the \textbf{classical master equation} (CME)
\begin{equation}
  \label{eq:cme}
  \{ S, S \} = 0.
\end{equation}
The data $(\Ecal, \langle \cdot, \cdot \rangle, S)$ defines a \textbf{classical BV theory}. The CME says $\{ S, \cdot \}$ is a differential which makes $(\Ocal ( \Ecal ), \{S, \cdot \} )$ into a cochain complex such that
\begin{equation*}
  \Hrm^0 \Ocal ( \Ecal ) \cong
  \Ocal( \mathrm{Crit} (S) ),
\end{equation*}
where $\mathrm{Crit} (S)$ denotes the critical locus of $S$. We will restrict to $S$ of the form
\begin{equation*}
S(e) = \underbrace{\langle e, Qe \rangle}_{\substack{ \text{free part} \\ \text{(kinetic +} \\ \text{mass terms)} }}
+ \underbrace{I(e)}_{\substack{ \text{interaction} \\ \text{part (cubic} \\ \text{or higher)} }}.
\end{equation*}

\begin{example}
  Why are the cubic and higher order terms called interaction terms? For electromagnetism on a manifold $M$ we have a space of fields
  $\Fcal = \Omega^1(M) \oplus \Omega^0(M, S)$ in degree $0$. Let $F = \drm A$ and define
  \begin{equation*}
    S(A, \psi) = \int_M
    \underbrace{F \wedge \hodge F
    + \langle \psi, \slashed{\drm} \psi \rangle \dd \mathrm{vol}}_{\text{quadratic terms}}
    + \underbrace{\langle \psi, \slashed{A} \psi \rangle \dd \mathrm{vol}}_{\text{interaction term}}.
  \end{equation*}
  Computing the Euler-Lagrange equations we obtain the system of differential equations
  \begin{equation*}
    \begin{cases}
      \hodge \drm \hodge F = \bar{\psi} \gamma^\mu \psi \dd x_\mu \\
      \slashed{d}_A \psi = 0
    \end{cases}
  \end{equation*}
  which is coupled because of the interaction term.
\end{example}

\section{Quantization in the BV formalism}

The slogan of quantization in the BV formalism is to \emph{deform the differential}. In the perturbative context we work in formal power series in $\hbar$, for example, over the ring $\Rbb[[\hbar]]$. Quantization results in a cochain complex
$(\Ocal(\Ecal)[[\hbar]], \{S^q, \cdot\} + \hbar \Delta)$, where $\Delta$ is called the BV Laplacian, and 
$S^q \in \Ocal(\Ecal)[[\hbar]]$ satisfies the \textbf{quantum master equation} (QME)
\begin{equation}
  \label{eq:qme}
  (\{S^q, \cdot \} + \hbar \Delta)^2 = 0
\end{equation}

\begin{example}
  In finite dimensions ($\Fcal \cong \Rbb^n$) the BV fields are
  $\Ecal = \Rbb^n \longrightarrow \Rbb^n$
  therefore
  \begin{equation*}
    \Ocal(\Ecal) \cong \Rbb [x^1, \dots, x^n, \xi^1, \dots, \xi^n]
  \end{equation*}
  and the BV Laplacian takes the form
  \begin{equation*}
    \Delta = \sum_{\mu = 1}^n \frac{\partial}{\partial \xi^\mu} \frac{\partial}{\partial x^\mu}.
  \end{equation*}
  In this form, it becomes clear that $\Delta$ is a differential operator of degree $1$ such that $\Delta^2 = 0$.
\end{example}
\begin{equation*}
  S^q(e) = \langle e, Qe \rangle
  + I^q(e)
\end{equation*}
where $I^q \in \Ocal(\Ecal) [[\hbar]]$ is cubic mod $\hbar$ and satisfies the QME
\begin{equation*}
  Q I^q + \frac{1}{2} \{I^q, I^q\} + \hbar \Delta I^q = 0
\end{equation*}
which resembles, in this form, the \textbf{Maurer-Cartan} (MC) \textbf{equation}. In infinite dimensions, some problems arise:

\begin{enumerate}
  \item there may be no solution to this equation. In this case we say that quantization is obstructed (there is an anomaly);
  \item the QME in infinite dimensions is ill-defined. Some functional analysis is needed to make sense of this problem.
\end{enumerate}

\chapter{Lecture 2}

In this lecture we consider a naive example that aims to exemplify how the Euler-Lagrange equations lead us to classical BV theories.

\begin{example}
  Let $\Fcal$ be a finite-dimensional vector space encoding the naive space of fields and consider an action
  \begin{equation*}
    S: \Fcal \longrightarrow \Rbb.
  \end{equation*}
  We say that $S$ is a naive action because it might be necessary to add additional terms to $S$ to guarantee that it satisfies the CME. The solutions to the Euler-Lagrange equation are fields $f \in \Fcal$ such that ${\drm S}_f = 0$. Restricting to the case $\Fcal = M$ for some finite-dimensional manifold $M$, we say that critical points of the action form the \textbf{critical locus} of $S$
  \begin{equation*}
    \Crit (S) = \bigl\{ p \in M \bigm| \drm S_p = 0 \bigr\}.
  \end{equation*}
  Alternatively, we can characterize the critical locus of $S$ as an intersection in $T^* M$
  \begin{equation*}
    \Crit (S) = \Graph( \drm S) \cap \Graph (M)
  \end{equation*}
  where we identify $M$ with the zero section. It follows that
  \begin{equation*}
    \Ocal( \Crit (S) ) =
    \Ocal( \Graph (\drm S ) ) \otimes_{\Ocal(T^* M)} \Ocal(M).
  \end{equation*}
  We are going to consider a derived version of this construction, where the tensor product $\otimes$ is replaced by a derived tensor product $\otimes^{\Lbb}$. This raises the obvious questions:
  \begin{figure}[ht]
    \begin{tikzpicture}
      \draw[thick, ->] (-4.5, 0) -- (4.5, 0);
      \draw[thick, domain=-3.37:3.37, smooth, variable=\x] plot ( {\x}, {.35*sin(2*deg(\x))*\x*\x-.35} );
      \draw[ultra thick, color=blue] (-3.1, 0) circle (5pt);
      \draw[ultra thick, color=blue] (-1.741, 0) circle (5pt);
      \draw[ultra thick, color=red] (1.12, 0) circle (5pt);
      \draw[ultra thick, color=blue] (3.18, 0) circle (5pt);
    \end{tikzpicture}
    \centering
    \caption{Well-behaved (in red) and badly-behaved (in red) points of an intersection.}
    \label{fig:bad_intersections}
  \end{figure}
  \begin{itemize}
    \item \textbf{Why?} This intersection might not be well behaved, in the sense that $\drm S$ and the zero section might not intersect transversally at every point, as illustrated in figure \ref{fig:bad_intersections}. The derived approach allows us to study these badly-behaved points using Serre's intersection formula.
    \item \textbf{How?} We replace $\Ocal (\Graph(\drm S)) \otimes_{\Ocal (T^*M)} \Ocal(M)$ with a dg commutative algebra $A$ such that
    \begin{equation*}
      \Hrm^0 A = \Ocal (\Graph(\drm S)) \otimes_{\Ocal (T^*M)} \Ocal(M).
    \end{equation*}
    To compute the derived tensor product $\otimes^{\Lbb}$ we need to resolve either $\Ocal(M)$ or $\Ocal(\Graph(\drm S))$ in $\Ocal(T^* M)$-modules. Let us make use of Darboux coordinates to resolve
    \begin{align*}
      \Ocal(\Graph(\drm S)) &=
      \bigslant{\Ocal(T^* M)}{ \bigl( f \vert_{\Graph(\drm S)} = 0 \bigr) } \\
                            &= \bigslant{ \Ocal( T^* M )}{ \bigl( p_{\mu} - \partial_\mu S \bigr) }.
    \end{align*}
    Consider the resolution
    \begin{equation*}
      \begin{tikzcd}[column sep=4.1em]
        \dots \arrow[r] &
        \Ocal(T^* M) (\xi_1, \dots, \xi_n) \arrow[r, "\xi_{\mu} \mapsto p_{\mu} - \partial_{\mu} S"] &
        \Ocal(T^* M) \arrow[r] &
        \Ocal(\Graph(\drm S))
      \end{tikzcd}
    \end{equation*}
    which we extend to the left as a Koszul complex
    $K^{-p} = \bigwedge_{\Ocal(T^* M)}^p (\xi_1, \dots, \xi_n)$
    with differential 
    \begin{equation*}
      \drm = \sum_{\mu} (p_{\mu} - \partial_{\mu} S) \frac{\partial}{\partial \xi_{\mu}}.
    \end{equation*}
    This complex freely resolves $\Ocal(\Graph(\drm S))$. Alternatively, $(K^\bullet, \drm)$ admits a coordinate free description where
    \begin{equation*}
      K^{-p} = \Ocal(T^* M) \otimes_{\Ocal(M)} \Xfrak^p (M).
    \end{equation*}
    A model for $\Ocal(\Graph(\drm S)) \otimes_{\Ocal(T^* M)}^{\Lbb} \Ocal(M)$ is given by
    \begin{equation*}
      \Ocal(\dCrit(S)) = K^{-\bullet} \otimes_{T^* M} \Ocal(M)
    \end{equation*}
    which we call the \textbf{derived critical locus}. But notice that
    \begin{equation*}
      \Ocal(T^* M) \otimes_{\Ocal(M)} \PV^\bullet(M) \otimes_{\Ocal(T^* M)} \Ocal(M) \cong \PV^\bullet(M)
    \end{equation*}
    where $\PV^\bullet(M)$ denotes the complex of polyvector fields on M. The differential is given by contracting with $\drm S$, so we write
    \begin{equation*}
      \Ocal(\dCrit(S)) = (\PV^\bullet(M), -\iota_{\drm S}).
    \end{equation*}
  \end{itemize}
\end{example}

\chapter{Lecture 3}

We want to sketch how to go from the Yang-Mills action
\begin{equation*}
  S^{\text{naive}} (A) = \int_{M^n} \tr (F_A \wedge \star F_A)
\end{equation*}
to the Yang-Mills classical BV theory
\begin{equation*}
  \begin{tikzcd}[sep=large]
    \underbrace{\Omega^0 (M, \gfrak)}_{\text{ghosts}}^{-1}
    \arrow[r, "\drm"] &
    \underbrace{\Omega^1 (M, \gfrak)}_{\text{fields}}^{0}
    \arrow[r, "\drm \star \drm"] &
    \underbrace{\Omega^{n-1} (M, \gfrak)}_{\text{antifields}}^{1}
    \arrow[r, "\drm"] &
    \underbrace{\Omega^n (M, \gfrak)}_{\text{antighosts}}^{2}
  \end{tikzcd}
\end{equation*}
with BV action
\begin{equation*}
  S^{\text{BV}}(e) = \langle e, Q e \rangle + I(e)
\end{equation*}
where
\begin{equation*}
  \langle e, f \rangle = \int_{M^n} \tr (e \wedge f)
\end{equation*}
is the $-1$-shifted symplectic structure. There are some points to motivate:
\begin{enumerate}[i)]
  \item \textbf{fields} $\longrightsquigarrow$ \textbf{fields and antifields:} coming from the derived critical locus $\text{dCrit}(S)$;
  \item \textbf{ghosts:} coming from taking the derived coinvariants of $\gfrak \curvearrowright V$.
\end{enumerate}

For Yang-Mills the manifold $M^n$ is spacetime and $\Omega^1(M, \gfrak)$ is the space of fields. In what follows, let $M$ be the space of fields. Recall that
\begin{align*}
  \Crit(S) &= \bigl\{ p \in M \bigm| \drm S_p = 0 \bigr\} \\
           &= \Graph (\drm S) \cap M
\end{align*}
in $T^* M$. Dually
\begin{equation*}
  \Ocal \bigl( \Crit(S) \bigr)
  = \Ocal \bigl( \Graph (\drm S) \bigr) \otimes_{\Ocal(T^* M)} \Ocal(M).
\end{equation*}
By homological yoga, taking the derived intersection means that we replace the tensor product $\otimes$ with the derived tensor product $\otimes^{\Lbb}$. To find $\dCrit (S)$ we resolve either $\Ocal \bigl( \Graph (\drm S) \bigr)$ or $\Ocal(M)$ as $\Ocal(T^* M)$-modules. Last time we wrote the Koszul complex
\begin{equation*}
  K^{-p} = \PV^p (M) \otimes_{\Ocal(M)} \Ocal (T^* M)
\end{equation*}
where $\PV^p = \bigwedge^p \Xfrak(M)$ and differential
\begin{equation*}
  Q : v_1 \wedge \dots \wedge v_k \otimes 1 \longmapsto
  \sum_{i=1}^k (-1)^{i+1} v_1 \wedge \dots \wedge \hat{v}_i \wedge \dots \wedge v_k
  \otimes \bigl( p(v_i) - \drm S(v_i) \bigr)
\end{equation*}

\begin{exercise}
  Check that $\Hrm^0 (K^{\bullet}, Q) \cong \Ocal \bigl( \Graph (\drm S) \bigr)$, so
  \begin{equation*}
    \dCrit (S) \cong K^{\bullet} \otimes_{\Ocal(T^* M)} \Ocal (M) \cong \PV^{\bullet} (M)
  \end{equation*}
  and thus $\Ocal \bigl( \dCrit(S) \bigr) \simeq (\PV^\bullet, - \iota_{\drm S})$.
\end{exercise}

\begin{exercise}
   Show that $\Hrm^0 \Ocal( \dCrit ) \cong \Ocal (\Crit)$.
\end{exercise}

We can enhance $\Ocal \bigl( \dCrit (S) \bigr)$ to a sheaf on $M$. Following Grothendieck
\begin{equation*}
  \dCrit(S) = \bigl( M, \PV_M^\bullet, - \iota_{\drm S} \bigr)
\end{equation*}
is an example of a \textbf{dg manifold}.

\begin{definition}
  A dg manifold is a smooth manifold $M$ with a sheaf $\Ocal_M$ of \textbf{dg commutative algebras} (DGCA) locally isomorphic to $\Ocal_M(U) \cong \bigwedge \Ecal (U)$ where $\Ecal$ are the smooth sections of $E \rightarrow M$.
\end{definition}

Ignoring the differential, we get a sheaf $(M, \PV_M^\bullet)$ on $M$ such that
\begin{equation*}
  \PV_M = \bigwedge \Xfrak_M \cong \Sym \Xfrak[1].
\end{equation*}
The underlying graded manifold of $\dCrit(S)$ is
\begin{equation*}
  T^*[-1]M = \bigl( M, \Sym \Xfrak[1] \bigr)
\end{equation*}
displaying the following properties:
\begin{enumerate}[i)]
  \item the graded manifold $T^*[-1]M$ is a $-1$-shifted symplectic graded manifold just as $T^*M$ is a $0$-shifted symplectic manifold;
  \item Induced from the $-1$-shifted symplectic structure we get a $1$-shifted Poisson bracket on $\Ocal \bigl( T^* [-1] M \bigr) = \PV(M)$ known as the \textbf{Schouten bracket}
  \begin{align*}
    \{ f, g \} &= 0, \\
    \{v, f \} &= v f, \\
    \{ v, w \} &= [v, w], \\
    \{u, v \wedge w \} &= \{u, v\} \wedge w + v \wedge \{ u, w \}
  \end{align*}
for $f, g \in \Ocal(M)$ and $u, v, w \in \PV^{-1}(M)$.
\end{enumerate}

\begin{exercise}
  Show that $-\iota_{\drm S} = \{ S, \cdot \}$.
\end{exercise}

\begin{definition}
  A $\Pbb_0$ algebra $\bigl( A, \drm, \{ \cdot, \cdot \} \bigr)$ is a DGCA $(A, \drm)$ with a $-1$-shifted Poisson bracket $\{ \cdot, \cdot \}: A \otimes A \rightarrow A$ obeying:
  \begin{enumerate}[i)]
    \item \textbf{graded skew-symmetry:}
      \begin{equation*}
        \{x, y\} = - (-1)^{(|x|+1)(|y|+1)} \{y, x\};
      \end{equation*}
    \item \textbf{graded Poisson identity}:
      \begin{equation*}
        \{x, yz \} = \{x, y\} z + (-1)^{(|x|+1)|y|} y \{x, z\}
      \end{equation*}
      so $\{x, \cdot \}$ is a degree $|x|+1$ derivation;
    \item \textbf{graded Jacobi identity}:
      \begin{equation*}
        \{x, \{y, z\} \} = \{ \{x, y\}, z \} + (-1)^{(|x|+1)(|y|+1)} \{y, \{x, z\} \};
      \end{equation*}
    \item \textbf{compatibility with differential:}
      \begin{equation*}
        \drm \{x, y\} = \{ \drm x, y \} + (-1)^{|x|+1} \{x, \drm y \}.
      \end{equation*}
  \end{enumerate}
\end{definition}

\begin{exercise}
  Check that the Schouten bracket defines a $\Pbb_0$ algebra on $\Ocal\bigl(T^*[-1] M \bigr)$.
\end{exercise}

\chapter{Lecture 20240429}

(John Huerta)

\section{Ultimate Goal}

Define and use the Feynman ``path'' integral
\begin{equation*}
  \int_{\phi \in \Fcal} e^{-\frac{S (\phi)}{\hbar}}
\end{equation*}
(Euclidean field theory)

In the constructive track: see Gonçalo on how to do this. In the BV
track: we will produce a formal power series in $\hbar$.

\section{Recall}

\begin{itemize}
\item From now on: We work {\em perturbatively}, i.e., {\em formally}
  (in Algebraic Geometry speak), i.e., in {\em formal power series},
  i.e., {\em infinitesimally}.
\item Now $M$ is going to be a finite dimensional manifold, denoting space-time. E.g.,
  \begin{align*}
    M&=R^d \intertext{or} \\
    M&=\mathrm{pt}
  \end{align*}
\item $\Fcal$ always denotes the naive fields, a sheaf of vector
  spaces on $M$; specifically, sections of some vector bundle
  $F\longrightarrow M$.

  Example: Yang-Mills fields for a trivial $G$-bundle
  $M\times G \longrightarrow M$, then
  $\Fcal (M)=\Omega^1 (M,\mathfrak{g})$, where
  $\mathfrak{g}=\mathrm{Lie} (G)$.
\item $\Ecal$ (``extended''), the space of BV-fields, a sheaf of {\em
    graded} vector spaces over $M$, sections of a graded vector bundle
  $E\longrightarrow M$. $\Ecal^0 (M)=\Fcal (M)$.

  In the Yang-Mills example, where $d=\mathrm{dim} M$
  \begin{equation*}
    \Ecal(M) = \underbrace{\Omega^0 (M,\mathfrak{g})}_{\text{``ghosts''}}
    \oplus\underbrace{\Omega^1(M,\mathfrak{g})}_{\text{``fields''}}
    \oplus\Omega^{d-1} (M,\mathfrak{g})
    \oplus\Omega^d(M,\mathfrak{g})
  \end{equation*}
\end{itemize}

\section{BV Formulation of Gauge Theory}

TODO

\section{Lightning Fast Introduction to Derived Invariants}

TODO

\section{Back to Yang-Mills}

TODO
\sloppy
\printbibliography
\end{document}
