\chapter{Lecture 20240429}

(John Huerta)

\section{Ultimate Goal}

Define and use the Feynman ``path'' integral
\begin{equation*}
  \int_{\phi \in \Fcal} e^{-\frac{S (\phi)}{\hbar}}
\end{equation*}
(Euclidean field theory)

In the constructive track: see Gonçalo on how to do this. In the BV
track: we will produce a formal power series in $\hbar$.

\section{Recall}

\begin{itemize}
\item From now on: We work {\em perturbatively}, i.e., {\em formally}
  (in Algebraic Geometry speak), i.e., in {\em formal power series},
  i.e., {\em infinitesimally}.
\item Now $M$ is going to be a finite dimensional manifold, denoting space-time. E.g.,
  \begin{align*}
    M&=\Rbb^d \intertext{or} \\
    M&=\mathrm{pt}
  \end{align*}
\item $\Fcal$ always denotes the naive fields, a sheaf of vector
  spaces on $M$; specifically, sections of some vector bundle
  $F\longrightarrow M$.

  Example: Yang-Mills fields for a trivial $G$-bundle
  $M\times G \longrightarrow M$, then
  $\Fcal (M)=\Omega^1 (M,\mathfrak{g})$, where
  $\mathfrak{g}=\mathrm{Lie} (G)$.
\item $\Ecal$ (``extended''), the space of BV-fields, a sheaf of {\em
    graded} vector spaces over $M$, sections of a graded vector bundle
  $E\longrightarrow M$. $\Ecal^0 (M)=\Fcal (M)$.

  In the Yang-Mills example, where $d=\mathrm{dim} M$
  \begin{equation*}
    \Ecal(M) = \underbrace{\Omega^0 (M,\mathfrak{g})}_{\text{``ghosts''}}^{-1}
    \oplus\underbrace{\Omega^1(M,\mathfrak{g})}_{\text{``fields''}}^{0}
    \oplus\underbrace{\Omega^{d-1} (M,\mathfrak{g})}_{\text{``anti-fields''}}^{1}
    \oplus\underbrace{\Omega^d(M,\mathfrak{g})}_{\text{``anti-ghosts''}}^{2}
  \end{equation*}
\end{itemize}

\section{BV Formulation of Gauge Theory}

Input: Naive gauge theory 
\begin{equation*}
\underbrace{\Lcal}_{
  \substack{\text{Lie algebra of}\\
    \text{infinitesimal gauge}\\
    \text{transformations}} } \curvearrowright
\underbrace{\Fcal}_{\text{space of naive fields}}\,.
\end{equation*}
The action may be non-linear. In the Young-Mills example it is an
affine action
\begin{equation*}
  \Omega^0 (M\mathfrak{g})\curvearrowright\Omega^1 (M\mathfrak{g})\,.
\end{equation*}
There is a two-step process to writing down the gauge theory:
\begin{enumerate}
\item Take the {\em ``stacky quotient''}
  \begin{equation*}
    \Fcal \rightsquigarrow \Fcal \sslash \Lcal
  \end{equation*}
\end{enumerate}

\section{Lightning Fast Introduction to Derived Invariants}

TODO

\section{Back to Yang-Mills}

TODO